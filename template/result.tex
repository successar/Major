\chapter{Results and Discussion}

\section{Data}
To compare our work with previously studied methods, we report
results on the widely used ATIS dataset [24, 25]. This
dataset is from the air travel domain and consists of audio
recordings of speakers making travel reservations. All
the words are labeled with a semantic label in a BIO format
(B: begin, I: inside, O: outside), e.g. New York contains
two words New and York and is therefore labeled with
B-fromloc.city name and I-fromloc.city name
respectively. Words which do not have semantic labels are
tagged with O. In total, the number of semantic labels is 127,
including the label of the class O. The training data consists of
4,978 sentences and 56,590 words. The test set contains 893
sentences and 9,198 words. To evaluate our models, we used the
script provided in the text chunking CoNLL shared task 20001
in line with other related work.

\section{Evaluation Metrics}
The most commonly used metrics for Intent Detection(ID) and Slot Filling(SF) are class (or slot)
error rate (ER) and F-Measure. The simpler metric ER for ID can
be computed as:
\begin{equation}
ER_{ID} = \frac{\text{\# misclassified utterances}}{\text{\# utterances}}
\end{equation}
Note that one utterance can have more than one intent. A typical
example is "Can you tell me my balance? I need to make a transfer."
In most cases, where the second intent is generic (a greeting, small
talk with the human agent) or vague, it is ignored. If none of the true
classes is selected, it is counted as a misclassification.
\par
For SF, the error rate can be computed in two ways: The more
common metric is the F-measure using the slots as units. This metric
is similar to what is being used for other sequence classification tasks
in the natural language processing community, such as parsing and
named entity extraction. In this technique, usually the IOB schema
is adopted, where each of the words are tagged with their position in
the slot: beginning (B), in (I) or other (O). Then, recall and precision
values are computed for each of the slots. A slot is considered to be
correct if its range and type are correct. The F-Measure is defined as
the harmonic mean of recall and precision:
\begin{eqnarray}
F1-Measure = \frac{2 \times \text{Recall} \times \text{Precision}}{\text{Recall} + \text{Precision}} \\
Recall = \frac{\text{\# correct slots found}}{\text{\# true slots}} \\
Precision = \frac{\text{\# correct slots found}}{\text{\# found slots}}
\end{eqnarray}

\section{Model Training}
We used LSTM cell as the basic RNN unit, following the LSTM design in (Zaremba et al., 2014).
The default forget gate bias was set to 1. We used single layer uni-directional LSTM in the proposed
joint online SLU-LM model. Deeper models by stacking the LSTM layers are to be explored in future work. Word embeddings of size 300 were randomly initialized and fine-tuned during model training. We conducted mini-batch training (with batch size 16) using Adam optimization method following the suggested parameter setup in (Kingma and Ba, 2014). Maximum norm for gradient clipping was set to 5. During model training, we applied dropout (dropout rate 0.5) to the non-recurrent connections (Zaremba et al., 2014) of RNN and the hidden layers of MLPs, and applied L2 regularization ($\gamma = 10^{-4}$) on the parameters of MLPs. 
\section{Results}
Table 1 summarises the result our method and its variations on ATIS dataset.
\begin{table}[h]
	\centering
\begin{tabular}{ |l|l|l| }
		\hline
		Model & Intent  Error & F1 Score \\ \hline
		RecNN (Guo et al., 2014) & 4.60 & 93.22 \\ \hline
		RecNN+Viterbi (Guo et al., 2014) & 4.60 & 93.96 \\ \hline
		Independent training RNN intent model & 2.13 & - \\ \hline
		Independent training RNN slot filling model & - & 94.91 \\ \hline
		Basic joint training model & 2.02 & 94.15 \\ \hline
		Joint model with recurrent intent context & 1.90 & 94.16 \\ \hline 
		Joint model with recurrent slot label context & 1.79 & 94.64 \\ \hline
		Joint model with recurrent intent + slot label context & 1.57 & 94.47 \\ \hline
		Independent Attention BiRNN & - & 95.4 \\ \hline
		Joint Attention BiRNN & 1.60 & 95.54 \\ \hline
		Joint Attention BiRNN (with Ranking Loss) & 1.61 & 95.75 \\ \hline
	\end{tabular}
	\caption[ATIS test set Results]{ATIS Test set results on intent detection error, slot filling F1 score, and language modeling perplexity. Related joint models: RecNN: Joint intent detection and slot filling model using recursive	neural network (Guo et al., 2014). RecNN+Viterbi: Joint intent detection and slot filling model using recursive neural network with Viterbi sequence optimization for slot filling (Guo et al., 2014).}
\end{table}
The basic joint model uses a shared representation for all the three tasks. It gives slightly
better performance on intent detection, with some degradation on slot
filling F1 score. If the RNN output $h_t$ is connected to each task output directly via linear projection without using MLP, performance drops for intent classification and slot filling. Thus, we believe the extra discriminative power introduced by the additional model parameters and non-linearity from MLP is useful for the joint model.
\par
As can be seen from the results, the joint model that utilizes two types of recurrent context maintains the benefits of both, namely, the benefit of applying recurrent intent
context to intent detection, and the benefit of applying recurrent slot label context to slot filling. Another observation is that once recurrent context is applied, the benefit of adding local context for next word prediction is limited. It might hint that the most useful information for the next word prediction can be well captured in the RNN state, and thus adding explicit dependencies on local intent class and slot label is not very helpful.
\par
For the attention-based bidirectional RNN architecture, the joint training model achieves 0.23\% absolute gain on slot filling over the independent training models. The attention-based
RNN model seems to benefit more from the joint training. Results from both of our joint training approaches outperform the best reported joint modeling results. Instead of applying a softmax layer and training the network with the cross entropy loss function, we also used the ranking loss
function. The hyper-parameters of the ranking loss function are optimized with a 5-fold cross validation. The best parameters are: $\gamma$ = 2, m+ = 3, m- = 0.5. We obtain 0.21\% absolute improvement compared to the cross entropy loss function

\section{Error Analysis}
\begin{table}[h]
	\centering
	\begin{tabular}{ |l|l|l|l|l|l|l|l|l|l|l|l|l|l|l|l|l|l| }
		\hline
		Correct-Estimated & a & b & b & d & e & f & g & h & i & j & k & l & m & n & o & p & q \\ \hline
		a. Abbreviation & 30 &  &  &  &  &  &  &  &  & 2 &  &  &  &  &  &  &  \\ \hline
		b. Aircraft &  & 6 &  &  &  &  &  &  &  & 3 &  &  &  &  &  &  &  \\ \hline
		c. Airfare &  &  & 64 &  &  &  &  &  &  & 1 &  &  &  &  &  &  &  \\ \hline
		d. Airline &  &  &  & 37 &  &  &  &  &  & 2 &  &  &  &  &  &  &  \\ \hline
		e. Airport &  &  &  &  & 15 &  &  &  &  & 2 &  &  &  &  & 1 &  &  \\ \hline
		f. Capacity & 1 & 5 &  &  &  & 13 &  &  &  & 2 &  &  &  &  &  &  &  \\ \hline
		g. City &  &  &  &  &  &  & 3 &  &  & 2 &  &  &  &  &  &  &  \\ \hline
		h. Day Name &  &  &  &  &  &  &  &  &  & 2 &  &  &  &  &  &  &  \\ \hline
		i. Distance &  &  &  &  &  &  &  &  & 9 & 1 &  &  &  &  &  &  &  \\ \hline
		j. Flight &  & 1 & 1 &  &  &  & 1 &  &  & 623 &  &  &  &  &  &  &  \\ \hline
		k. Flight No &  &  &  &  &  &  &  &  &  & 2 & 6 &  &  &  &  &  &  \\ \hline
		l. Flight Time &  &  &  &  &  &  &  &  &  & 1 &  &  &  &  &  &  &  \\ \hline
		m. Ground Fare &  &  & 1 &  &  &  &  &  &  &  &  &  & 3 & 3 &  &  &  \\ \hline
		n. Ground Service &  &  &  &  &  &  &  &  &  &  &  &  &  & 36 &  &  &  \\ \hline
		o. Meal &  &  &  &  &  &  &  &  &  & 5 &  &  &  &  &  &  &  \\ \hline
		p. Quantity &  &  &  &  &  &  &  &  &  &  &  &  &  &  &  & 8 &  \\ \hline
		q. Restriction & 1 &  &  &  &  &  &  &  &  &  &  &  &  &  &  &  &  \\ \hline
	\end{tabular}
	\caption{Confusion Matrix for Intent Detection}
\end{table}
\subsection{Intent Detection Errors}
The problem is mostly the non-Flight utterances erroneously classified as Flight. While one cause of these errors is the unbalanced intent distribution, we have manually checked
each error and clustered them into 6 categories:
\begin{enumerate}
\item \textbf{Prepositional phrases embedded in noun phrases:} These errors
involve phrases such as Capacity of the flight from Boston
to Orlando, where the prepositional phrase suggests flight information,
whereas the destination category is mainly determined
by the head word of the noun phrase (capacity in this
case). Since classifier has no syntactic features, such sentences
are usually classified erroneously. Using features from
a syntactic parser can alleviate this problem.
\item \textbf{Wrong functional arguments of utterances:} This category is
similar to the first category but the difference is that, instead
of a prepositional phrase, the confused phrase is a semantic
argument of the utterance. Consider the example utterance
What day of the week does the flight from Boston to Orlando
fly? These are errors that can be solved by using either a syntactic
parser that identifies functions of phrases or a semantic
role labeler.
\item \textbf{Annotation errors: }These are utterances that were assigned
the wrong category during manual annotation.
\item \textbf{Utterances with multiple sentences: }These are utterances with
more than one sentence. In such cases, the intent is usually in
the last sentence, whereas the classification output is biased
by the other sentence.
\item \textbf{Other:} These include several infrequent error types such as
ambiguous utterances, ill-formulated queries, and preprocessing/tokenization
issues:
\begin{itemize}
\item Ambiguous utterances: These errors involve utterances
where the destination category is not clear in the utterance.
An example from the ATIS test set is list Los
Angeles. In this utterance, the speaker intent could either
be to find cities that have flights from Los Angeles
or flights to Los Angeles.
\item Ill-formulated queries: These are utterances which include
a phrase that may mislead the classification or
understanding. An example from the ATIS test set is:
What’s the airfare for a taxi to the Denver airport? In
this case, the word airfare implies a destination category
of Airfare, whereas what is meant is Ground transportation
fare. These type of errors are easier for humans
to handle, but it is not presently clear how they
can be resolved in automatic processing.
\item Preprocessing/Tokenization issues: These are errors that
could be resolved by using a domain ontology or special
pre-processing or tokenization related to the domain.
Some domain specific abbreviations and restriction
codes are examples of this category.
\end{itemize}
\item \textbf{Difficult Cases:} These are utterances that include words or
phrases that were previously unseen in the training data. For
the example utterance Are snack served on Tower Air?, none
of the content words and phrases appear with the Meal category
in the training data.

\end{enumerate}

\subsection{Slot Filling Errors}
Analyzing the SF decisions, the model found 2,614 of 2,837 slots
with the correct type and span for the input out of 9,164 words.
We manually checked each of the 223 erroneous cases and clustered
them into 8 categories:
\begin{enumerate}
\item \textbf{Long distance dependencies:} These are slots where the disambiguating
tokens are out of the current n-gram context. For
example, in the utterance Find flights to New York arriving in
no later than next Saturday, a 6-gram context is required to
resolve that Saturday is the arrival date. 
\item \textbf{Partially correct slot value annotations:} These are slots assigned
a category that is partially correct; either the category
or the sub-category matches the manual annotation. For example,
the word tomorrow can either be a Depart Date.Relative
or Arrive Date.Relative for the utterance flights arriving in
Boston tomorrow. Note that these can overlap with other error
types.
\item \textbf{Previously unseen sequences:} While this category requires
further analysis, the most common reason is the mismatch
between the training and test sets. For example, meal related
slots are missed by the model (8.0\% of all errors) because
there are no similar cases in the training set. This is also the
case for the aircraft models (10.0\%), and traveling to states
instead of cities (3.3\%), etc.
\item \textbf{Annotation errors:} These are the slots that were assigned the
wrong category during manual annotation.
\item \textbf{Other:} These include several infrequent error types such as
ambiguous utterances, ill-formulated queries, and preprocessing/tokenization
issues:
\begin{itemize}
\item Ill-formulated queries: These errors usually involve an
ungrammatical phrase that may mislead the interpretation
of the slot value or there is insufficient context to
disambiguate the value of the slot. For example, in the
utterance Find a flight from Memphis to Tacoma dinner,
it is not clear if the word dinner refers to the description
of the flight meal.
\item Ambiguous utterances: These are utterances where the
slot category is not explicit given the utterance. For
example, in the utterance I would like to have the airline
that flies Toronto, Detroit and Orlando, it is not
clear if the speaker is searching for airlines that have
flights from Toronto to Detroit and Orlando or from
some other location to Toronto, Detroit and Orlando.
\item Preprocessing/Tokenization issues: These are errors that
could be resolved using a domain ontology or special
pre-processing or tokenization related to the domain.
For example, in the utterance What airline is AS, it
would be helpful to know AS is a domain specific abbreviation.
\item Ambiguous part-of-speech tag-related errors: These are
errors that could be resolved if the part-of-speech tags
were resolved. For example, the word arriving can be a
verb or an adjective, as in the utterance I want to find the
earliest arriving flight to Boston. In this case, the slot
category for the words earliest arriving is Flight-Mod,
but since the word arriving is very frequently seen as a
verb in this corpus, it is assigned no slot category.
\end{itemize}
\end{enumerate}